\documentclass[letterpaper,11pt]{article}

\usepackage{latexsym}
\usepackage[empty]{fullpage}
\usepackage{titlesec}
\usepackage{marvosym}
\usepackage[usenames,dvipsnames]{color}
\usepackage{verbatim}
\usepackage{enumitem}
\usepackage[pdftex]{hyperref}
\usepackage{fancyhdr}
\usepackage[utf8x]{inputenc}
\usepackage[ukrainian]{babel}

\pagestyle{fancy}
\fancyhf{} % clear all header and footer fields
\fancyfoot{}
\renewcommand{\headrulewidth}{0pt}
\renewcommand{\footrulewidth}{0pt}

% Adjust margins
\addtolength{\oddsidemargin}{-0.375in}
\addtolength{\evensidemargin}{-0.375in}
\addtolength{\textwidth}{1in}
\addtolength{\topmargin}{-.5in}
\addtolength{\textheight}{1.0in}

\urlstyle{same}

\raggedbottom
\raggedright
\setlength{\tabcolsep}{0in}

% Sections formatting
\titleformat{\section}{
  \vspace{-4pt}\scshape\raggedright\large
}{}{0em}{}[\color{black}\titlerule \vspace{-5pt}]

%-------------------------
% Custom commands
\newcommand{\resumeItem}[2]{
  \item\small{
    \textbf{#1}{: #2 \vspace{-2pt}}
  }
}

\newcommand{\resumeSubheading}[4]{
  \vspace{-1pt}\item
    \begin{tabular*}{0.97\textwidth}{l@{\extracolsep{\fill}}r}
      \textbf{#1} & #2 \\
      \textit{\small#3} & \textit{\small #4} \\
    \end{tabular*}\vspace{-5pt}
}

\newcommand{\resumeSubItem}[2]{\resumeItem{#1}{#2}\vspace{-4pt}}

\renewcommand{\labelitemii}{$\circ$}

\newcommand{\resumeSubHeadingListStart}{\begin{itemize}[leftmargin=*]}
\newcommand{\resumeSubHeadingListEnd}{\end{itemize}}
\newcommand{\resumeItemListStart}{\begin{itemize}}
\newcommand{\resumeItemListEnd}{\end{itemize}\vspace{-5pt}}

%-------------------------------------------
%%%%%%  CV STARTS HERE  %%%%%%%%%%%%%%%%%%%%%%%%%%%%


\begin{document}

%----------HEADING-----------------
\begin{tabular*}{\textwidth}{l@{\extracolsep{\fill}}r}
  \textbf{\href{https://www.linkedin.com/in/xairaven}{\Large Олександр Ковальов}} & Локація: Кропивницький або Київ \\
  \href{https://www.linkedin.com/in/xairaven}{{www.linkedin.com/in/xairaven}} & Пошта: \href{mailto:oleksandr.kovalov.work@gmail.com}{oleksandr.kovalov.work@gmail.com} \\ & Телефон: \href{tel:+380957365160}{+38 095 736 51 60} \\ &
  GitHub: \href{https://www.github.com/xairaven}{github.com/xairaven}
\end{tabular*}

%--------PROGRAMMING SKILLS------------
\section{Навички}
\resumeSubHeadingListStart
\item{
	\textbf{Англійська}{: B1 \textbf{(Говоріння \& Писання)}, B2-C1 \textbf{(Читання \& Розуміння)} }
}
\item{
	Rust \textit{(Основна мова)}, C\# \textit{(WPF)}, Java, C, Bash, Docker, Linux \textit{(Ubuntu Server, CentOS, Oracle Server)}, OOP, SOLID, SQL \textit{(SQLite, MS SQL, MySQL, MariaDB)}, Python, Apache, Networking, Git, WASM, Algorithms \& Data Structures, UML, Github Actions.
}
\resumeSubHeadingListEnd

%-----------EDUCATION-----------------
\section{Освіта}
  \resumeSubHeadingListStart
    \resumeSubheading
      {Київський політехнічний інститут ім. Ігоря Сікорського}{Київ}
      {Бакалавр, Комп'ютерні науки -- Цифрові технології в енергетиці}{Вер. 2021 -- Чер. 2025}
    \resumeSubheading
    	{Київський політехнічний інститут ім. Ігоря Сікорського}{Київ}
    	{Магістратура \textbf{(Навчаюсь)}, Інженерія програмного забезпечення}{Вер. 2025 -- \textbf{Чер. 2027}}
  \resumeSubHeadingListEnd

%-----------EXPERIENCE-----------------
\section{Досвід}
\resumeSubHeadingListStart

\resumeSubheading
{KPI-Craft}{Київ}
{Системний адміністратор, СЕО}{Жов. 2024 - Зараз}
\resumeItemListStart
\resumeItem{Системне адміністрування}
	{Розгорнув Linux-сервер, забезпечив моніторинг, резервне копіювання та налаштував систему плагінів.}
\resumeItem{CEO}
	{Заснував проєкт, займався підтримкою користувачів та його просуванням.}
\resumeItem{Безпека}
	{Успішно відбив кілька атак на інфраструктуру.}
\resumeItemListEnd

\resumeSubHeadingListEnd

%-----------PROJECTS-----------------
\section{Проекти}
  \resumeSubHeadingListStart
    \resumeSubItem{xailyser -- Rust, libpcap, nom, WebSocket, egui}
      {\\ Модульна платформа для глибокого аналізу пакетів (DPI), написана на Rust. Включає: основну бібліотеку DPI для захоплення пакетів та парсингу протоколів (libpcap + nom); багатопотоковий бекенд-сервер, що агрегує статистику трафіку та передає JSON-фрейми через WebSockets; кросплатформний клієнт на egui, який відображає графіки в реальному часі, псевдоніми пристроїв, пошук виробника за OUI та деталі пакетів. Проект демонструє експертизу в системному Rust, мережевих протоколах, багатопоточності та розробці UI. \\
      \textbf{Посилання}:  \href{https://github.com/xairaven/xailyser}{github.com/xairaven/xailyser}
	}
	\resumeSubItem{xArpChat -- Rust, Networks}
	{\\ Легкий peer-to-peer чат, написаний на Rust, що використовує ARP (Address Resolution Protocol) для комунікації в локальній мережі. Він підтримує широкомовну передачу повідомлень без необхідності в централізованому сервері, демонструючи володіння низькорівневими мережевими технологіями та моделлю конкурентності Rust. Як бібліотеку TUI використано Cursive. Також використані бібліотеки PNet, Crossbeam, smaz, thiserror. Створено workflow Github Actions для збірки та створення релізів. \\
		\textbf{Посилання}:  \href{https://github.com/xairaven/arpchat-rs}{github.com/xairaven/arpchat-rs}
	}
	\resumeSubItem{FractalRenderer -- Rust, WebAssembly}
	{\\ Rust, WebAssembly Інструмент візуалізації на основі Rust для генерації фракталів IFS (системи ітерованих функцій) та L-систем. Спочатку це були два окремі університетські лабораторні проєкти, об'єднані в єдину кодову базу. Пропонує як нативний GUI через egui, так і рендеринг у браузері через WebAssembly, що підкреслює досвід роботи з фронтенд-екосистемами Rust та кросплатформним розгортанням. \\
		\textbf{Посилання}: \href{https://github.com/xairaven/FractalRenderer}{github.com/xairaven/FractalRenderer}
	}
	\resumeSubItem{Blood Bank Management -- Linux, Apache, Docker, Ansible \textit{(Не мій проєкт, лише розгортання)}}
	{\\ Система управління банком крові з використанням PHP, MySQL, HTML, CSS3 + Bootstrap. Розгорнуто на двох локальних віртуальних машинах у кількох варіантах: ручний запуск Apache на першій VM та створення БД MySQL на другій; варіант з Docker + Docker-Compose; налаштування конфігурації за допомогою Ansible. \\
		\textbf{Посилання}:  \href{https://github.com/xairaven/BBMS-Deploy}{github.com/xairaven/BBMS-Deploy}
	}
  \resumeSubHeadingListEnd
  
  %-----------Certificates-----------------
\section{Сертифікати}
  \resumeSubHeadingListStart
    \resumeSubItem{Linux \& Network Administration}
	    {PortaOne, 2024}
     \resumeSubItem{DevOps Crash Course}
	    {SoftServe, 2023}
	 \resumeSubItem{English Certificate 71/100}
	 	{EF Set, Червень 2023}
  \resumeSubHeadingListEnd

%-------------------------------------------
\end{document}
